\documentclass[../main.tex]{memoir}

\begin{document}
\thispagestyle{empty}

\begin{center}
  {\large\bfseries \ProjectTitleEng}\\
\end{center}

\begin{center}
  \AuthorName\\
  \vspace{0.7cm}
  \noindent{\textbf{Keywords}:}\\
  Deep learning, crowd analysis, anomaly detection, video-surveillance\\
  \vspace{0.7cm}
  \noindent{\textbf{Abstract}}\\
\end{center}

Last decades have experimented and unprecedented growth in population
all around the globe. Crime and terrorism rates are also increasing
rapidly. This two issues have converted video-surveillance into a
fundamental tool worldwide. The number of security cameras installed
both at public and private environments is huge nowadays, and thus
processing the information retrieved by those cameras is getting
harder everyday. Then, there is a need of automation of that process,
using intelligent models capable of extracting information from video
sources.\\

Advances in deep learning produced in the last years have improved the
results obtained in this research area by a large margin. Current
models are able to extract complex information automatically, and to
work in more difficult environments, specially in terms of density of
individuals. Despite this recent advances, the area is still
relatively modern, and consequently is not properly structured. Due to
this fact, comparing works that tackle different sub-tasks within this
area is usually difficult.\\

In the previous context, this work tries to solve different tasks
related to the automatic treatment of video-surveillance sources. In
particular, we will focus our efforts in the analysis of crowd
behaviors from video-surveillance sources. The work is divided in two
parts. In the first part, which is the theoretical study, a sequential
taxonomy is proposed. This taxonomy following a sequence allows to
organize the different sub-tasks inside the topic as stages of a
pipeline. The results of the latter stages of the pipeline strongly
rely on the previous ones, and thus its results are heavily influenced
by the first models.\\

Apart from the proposed taxonomy, an in-depth review of the
state-of-the-art is conducted. Particularly, we center our study in
the works that use deep learning models to solve the crowd anomaly
detection problem. Inside this topic, we analyze the main public
datasets and proposals, organizing them depending on the specific type
of anomaly to be detected.\\

Finally, in the experimental part of this work, the use of
spatio-temporal features for action anomaly detection is studied. To
this end, one of the retrieved works is deeply analyzed and set as
baseline for comparison. After the complete study, a new model for
action anomaly detection is proposed. In particular, the original
model employs a 3D convolutional feature extractor, which processes
batches of consecutive frames. In our approach, the feature extractor
proposed is a combination of 2D CNNs for spatial feature extraction,
processing frames independently, together with a recurrent neural
network, which learns to process the sequence of features from
consecutive frames in the video. Our hypothesis defends that the
combination of convolutions and recurrent networks better preserves
the semantic structure of information in the video, and thus the
obtained model extracts more meaningful information.\\

Our experimentation suggests that our model outperforms the
classification capability of the original model, despite being
pretrained on a much smaller dataset. In particular, the original
feature extractor is trained on a set 1000 times bigger than ours.
This fact allows us to conclude that our proposal is better than
the original one in terms of classification capability, validating
our hypothesis.\\

The code developed for experimentation, together with the instructions
to replicate the experiments, can be found in the repository
\url{https://github.com/fluque1995/tfm-anomaly-detection}.

\newpage
\end{document}

%%% Local Variables:
%%% mode: latex
%%% TeX-master: "../main"
%%% End:
